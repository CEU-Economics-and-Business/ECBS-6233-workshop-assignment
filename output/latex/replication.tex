\documentclass[11pt]{article}
%\usepackage{booktabs}
%\usepackage{caption}
\usepackage{float}
\usepackage{adjustbox}

\usepackage{geometry}                		% See geometry.pdf to learn the layout options. There are lots.
\geometry{letterpaper}                   		% ... or a4paper or a5paper or ... 
%\geometry{landscape}                		% Activate for for rotated page geometry
%\usepackage[parfill]{parskip}    		% Activate to begin paragraphs with an empty line rather than an indent
\usepackage{graphicx}				% Use pdf, png, jpg, or eps§ with pdflatex; use eps in DVI mode
								% TeX will automatically convert eps --> pdf in pdflatex		
\usepackage{amssymb}

\title{Replication: Currie et al.: Fast-Food, Obesity, Weight Gain}
\date{}							% Activate to display a given date or no date

\begin{document}
\maketitle
%\section{}
%\subsection{}
\newpage
\begin{table}[H]
\caption{\label{fig: sum_stats} Summary Statistics for California School Data}
\vspace{-0.3cm}

\begin{center}\tiny
{
\def\sym#1{\ifmmode^{#1}\else\(^{#1}\)\fi}
\begin{tabular}{l*{4}{c}}
\hline\hline
                    &\multicolumn{1}{c}{All}&\multicolumn{1}{c}{$<$0.5 miles FF}&\multicolumn{1}{c}{$<$0.25 miles FF}&\multicolumn{1}{c}{$<$0.1 miles FF }\\
\hline
Number of obese students&     366.275&     384.300&     383.048&     400.746\\
\\ \textit{School Characteristics}&            &            &            &            \\
School qualified for title 1 funding&       0.397&       0.411&       0.406&       0.436\\
Number of students  &    1566.184&    1663.978&    1663.624&    1715.707\\
Student teacher ratio&      22.393&      22.841&      22.668&      22.857\\
Share black students&       0.084&       0.093&       0.093&       0.086\\
Share asian students&       0.107&       0.117&       0.118&       0.116\\
Share hispanic students&       0.380&       0.409&       0.416&       0.436\\
Share native american students&       0.014&       0.010&       0.010&       0.012\\
Share immigrant students&       0.034&       0.029&       0.030&       0.033\\
Share female students&       0.475&       0.477&       0.477&       0.490\\
Share eligible for free lunch&       0.290&       0.306&       0.313&       0.311\\
Share eligible for subsidised lunch&       0.063&       0.064&       0.063&       0.063\\
FTE teachers per student&       0.048&       0.047&       0.047&       0.047\\
Average test scores for 9th grade&      56.255&      54.964&      54.737&      52.291\\
Test score information missing&       0.024&       0.020&       0.022&       0.016\\
 \\  \textit{School District Characteristics}&            &            &            &            \\
Students teacher ratio&      20.897&      21.095&      20.911&      21.092\\
Share immigrant students&       0.028&       0.025&       0.025&       0.028\\
Share non-English speaking students&       0.206&       0.224&       0.225&       0.222\\
Share IEP students  &       0.126&       0.125&       0.132&       0.120\\
Staff student ratio &       0.102&       0.099&       0.102&       0.095\\
Share diploma recipients&       0.086&       0.084&       0.082&       0.091\\
Share diploma recipients missing&       0.004&       0.004&       0.008&       0.000\\
\\  \textit{Census Demographics} &            &            &            &            \\
Median household income in 1999&   48596.074&   45686.657&   44183.090&   44691.683\\
Median earnings in 1999 for popn >=16 with earnings&   25674.202&   24668.101&   24271.405&   23942.224\\
Average household size for all occupied housing units&       2.967&       2.932&       2.840&       2.881\\
Median contract rent (dollars) for specified renter occupied housing&     743.736&     741.186&     734.136&     706.286\\
Median gross rent (dollars) for specified renter occupied housing&     835.741&     825.461&     812.538&     781.927\\
Median Value for all Owner Occupied Housing Units&  202783.236&  199823.654&  199834.384&  195244.011\\
Percent of the population that is white&       0.629&       0.597&       0.591&       0.578\\
Percent of the population that is white&       0.056&       0.062&       0.064&       0.053\\
Percent of the population that is asian&       0.090&       0.099&       0.098&       0.110\\
Percent of the population that is male&       0.491&       0.489&       0.487&       0.494\\
Percent of the population 15+ never married&       0.289&       0.310&       0.320&       0.314\\
Percent of the population 15+ married (spouse present or absent)&       0.546&       0.519&       0.505&       0.513\\
Percent of the population divorced&       0.103&       0.107&       0.111&       0.107\\
Percent of popn 25+ with just a high school diploma&       0.220&       0.219&       0.219&       0.220\\
Percent of popn 25+ with some college, no degree&       0.235&       0.226&       0.223&       0.219\\
Percent of popn 25+ with Associate's degree&       0.072&       0.069&       0.071&       0.072\\
Percent of popn 25+ with Bachelor's degree&       0.150&       0.149&       0.151&       0.139\\
Percent of popn 25+ with Graduate degree&       0.078&       0.076&       0.077&       0.069\\
Percent of popn 16+ in labor force&       0.616&       0.618&       0.619&       0.617\\
Percent of popn 16+ in labor force that is unemployed&       0.083&       0.085&       0.088&       0.079\\
Percent of households with income under 10k&       0.092&       0.099&       0.106&       0.101\\
Percent of households with income over 200k&       0.026&       0.022&       0.020&       0.020\\
Percent of households with wage or salary income&       0.782&       0.784&       0.784&       0.789\\
Percent of housing units occupied&       0.946&       0.953&       0.953&       0.950\\
Percent of population in owner-occupied units&       0.592&       0.530&       0.481&       0.488\\
Percent of housing units considered urban&       0.912&       0.974&       0.971&       0.987\\
\\  \textit{Outcome Variable} &            &            &            &            \\
Percent obese students&      32.949&      33.772&      33.724&      35.733\\
\hline\hline
\end{tabular}
}


\end{center}
\vspace{-0.3cm}
\par
\begin{minipage}{ \linewidth}
\scriptsize{Notes: This table lists all the controls used for the school regressions, with the exception of year and school fixed
effects.}
\end{minipage}

\end{table}



\begin{table}[H]
\caption{\label{fig: sum_stats} Summary Statistics for California School Data}
\vspace{-0.3cm}

\begin{center}\small
{
\def\sym#1{\ifmmode^{#1}\else\(^{#1}\)\fi}
\begin{tabular}{l*{4}{c}}
\hline\hline
                    &\multicolumn{4}{c}{Percent of ninth graders that are obese }                           \\
                    &\multicolumn{1}{c}{(1)}         &\multicolumn{1}{c}{(2)}         &\multicolumn{1}{c}{(3)}         &\multicolumn{1}{c}{(4)}         \\
\hline
Fast food within 0.10 miles&       3.081\sym{*}  &       1.737\sym{**} &       6.270\sym{**} &       6.334\sym{**} \\
                    &     (1.607)         &     (0.880)         &     (2.918)         &     (2.875)         \\
[1em]
Other restaurant within 0.10 miles&       0.682         &      -0.631         &       1.026         &       1.003         \\
                    &     (1.031)         &     (0.577)         &     (1.820)         &     (1.824)         \\
[1em]
Fast food within 0.25 miles&      -2.486\sym{**} &      -0.907\sym{*}  &      -1.830         &      -1.795         \\
                    &     (1.111)         &     (0.548)         &     (1.197)         &     (1.210)         \\
[1em]
Other restaurant within 0.25 miles&       2.142\sym{**} &       0.041         &       0.262         &       0.037         \\
                    &     (0.876)         &     (0.493)         &     (0.972)         &     (0.943)         \\
[1em]
Fast food within 0.5 miles&       1.390\sym{*}  &      -0.045         &      -1.089         &      -0.831         \\
                    &     (0.822)         &     (0.451)         &     (1.109)         &     (1.087)         \\
[1em]
Other restaurant within 0.5 miles&       1.227         &       0.539         &      -0.397         &      -0.415         \\
                    &     (0.841)         &     (0.493)         &     (0.871)         &     (0.816)         \\
\hline
Implied cumulative effect \\ to fast food restaurant within 0.1 miles&      1.9852         &       .7854         &      3.3519         &      3.7079         \\
 (standard errors)  &       1.509         &         .86         &       3.037         &       2.984         \\
School FE           &                     &                     &  \checkmark         &  \checkmark         \\
Year FE             &                     &  \checkmark         &                     &  \checkmark         \\
Census Controls     &                     &  \checkmark         &                     &  \checkmark         \\
School Controls     &                     &  \checkmark         &                     &  \checkmark         \\
N                   &        8373         &        8373         &        8373         &        8373         \\
R2                  &       0.021         &       0.417         &       0.644         &       0.651         \\
\hline\hline
\end{tabular}
}


\end{center}
\vspace{-0.3cm}
\par
\begin{minipage}{ \linewidth}
\scriptsize{Notes: Each column is a different OLS regression. The regressions are weighted by the number of students. The
dependent variable is the percentage of students in the ninth grade who are classified as obese. The mean of the
dependent variable is 32.9494. The unit of observation is a school-grade-year for schools in California in 1999 and
the period 2001 -- 2007. Entries in rows 1, 3, and 5 are the coefficient on a dummy for the existence of a fast food
restaurant at a given distance from the school. Entries in rows 2, 4, and 6 are coefficients on the dummy for the existence
of a non-fast food restaurant at a given distance from the school. The implied cumulative effect reported in the
table is the sum of the coefficients in rows 1, 3, and 5, and is the total effect of exposure to a fast food restaurant at
0.1 mile compared to no exposure to fast food restaurants within 0.5 miles. The school-level controls are from the
Common Core of Data, with the addition of Star test scores for the ninth grade. The census block controls are from
the closest block to the address of the school. Table 1A lists the school and census block controls. Standard errors
clustered by school are in parentheses. \\\
 *** Significant at the 1 percent level,\\\
  ** Significant at the 5 percent level, \\\
  * Significant at the 10 percent level.}
\end{minipage}

\end{table}



\begin{figure}[H]
\vspace{0.5cm}
\begin{center}
\caption{\label{fig: obes_schools} Impact of Fast Food on Obesity in Schools}
\includegraphics[width=130 mm, height=90 mm]{../figures/figure1A.png} 
\end{center}
\vspace{-0.3cm}
\par
\begin{minipage}{ \linewidth}
\scriptsize{Notes: The blue vertical bar represent the 95 \% confidence interval using panel estimates; the red dashed vertical bar represent the 95 \% confidence interval using cross-sectional estimates}
\end{minipage}

\end{figure}


\begin{table}[H]
%\resizebox{5cm}{!}{

\caption{\label{fig: low_readability} A table not easy to read}
\vspace{-0.3cm}
\begin{center}

{
\def\sym#1{\ifmmode^{#1}\else\(^{#1}\)\fi}
\begin{tabular}{l*{4}{c}}
\hline\hline
                    &\multicolumn{4}{c}{Percent of ninth graders that are obese }                           \\
                    &\multicolumn{1}{c}{(1)}         &\multicolumn{1}{c}{(2)}         &\multicolumn{1}{c}{(3)}         &\multicolumn{1}{c}{(4)}         \\
\hline
\_Iyear\_2001         &                     &                     &       0.206         &      -0.089         \\
                    &                     &                     &     (0.653)         &     (0.664)         \\
[1em]
\_Iyear\_2002         &                     &                     &       3.355\sym{***}&       3.011\sym{***}\\
                    &                     &                     &     (0.717)         &     (0.752)         \\
[1em]
ffood1              &       3.081\sym{*}  &       1.737\sym{**} &       6.270\sym{**} &       6.334\sym{**} \\
                    &     (1.607)         &     (0.880)         &     (2.918)         &     (2.875)         \\
[1em]
ffood2              &      -2.486\sym{**} &      -0.907\sym{*}  &      -1.830         &      -1.795         \\
                    &     (1.111)         &     (0.548)         &     (1.197)         &     (1.210)         \\
[1em]
ffood3              &       1.390\sym{*}  &      -0.045         &      -1.089         &      -0.831         \\
                    &     (0.822)         &     (0.451)         &     (1.109)         &     (1.087)         \\
[1em]
\_Iyear\_2003         &                     &                     &       1.590\sym{**} &       0.805         \\
                    &                     &                     &     (0.652)         &     (0.711)         \\
[1em]
\_Iyear\_2004         &                     &                     &       1.711\sym{***}&       1.092         \\
                    &                     &                     &     (0.653)         &     (0.722)         \\
[1em]
\_Iyear\_2005         &                     &                     &       2.141\sym{***}&       1.201         \\
                    &                     &                     &     (0.674)         &     (0.766)         \\
[1em]
\_Iyear\_2006         &                     &                     &       1.042         &      -0.034         \\
                    &                     &                     &     (0.661)         &     (0.795)         \\
[1em]
\_Iyear\_2007         &                     &                     &       0.117         &      -1.041         \\
                    &                     &                     &     (0.632)         &     (0.777)         \\
[1em]
afood1              &       0.682         &      -0.631         &       1.026         &       1.003         \\
                    &     (1.031)         &     (0.577)         &     (1.820)         &     (1.824)         \\
[1em]
afood2              &       2.142\sym{**} &       0.041         &       0.262         &       0.037         \\
                    &     (0.876)         &     (0.493)         &     (0.972)         &     (0.943)         \\
[1em]
afood3              &       1.227         &       0.539         &      -0.397         &      -0.415         \\
                    &     (0.841)         &     (0.493)         &     (0.871)         &     (0.816)         \\
\hline
Observations        &        8373         &        8373         &        8373         &        8373         \\
\hline\hline
\end{tabular}
}



\vspace{-0.3cm}
\end{center}

\par
\begin{minipage}{ \linewidth}
\scriptsize{Notes: To be added.}
\end{minipage}
%}
\end{table}

\end{document}  